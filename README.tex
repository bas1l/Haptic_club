% !TEX TS-program = pdflatex
% !TEX encoding = UTF-8 Unicode

% This is a simple template for a LaTeX document using the "article" class.
% See "book", "report", "letter" for other types of document.

\documentclass[11pt]{article} % use larger type; default would be 10pt

\usepackage[utf8]{inputenc} % set input encoding (not needed with XeLaTeX)

%%% Examples of Article customizations
% These packages are optional, depending whether you want the features they provide.
% See the LaTeX Companion or other references for full information.

%%% PAGE DIMENSIONS
\usepackage{geometry} % to change the page dimensions
\geometry{a4paper} % or letterpaper (US) or a5paper or....
% \geometry{margin=2in} % for example, change the margins to 2 inches all round
% \geometry{landscape} % set up the page for landscape
%   read geometry.pdf for detailed page layout information

\usepackage{graphicx} % support the \includegraphics command and options

% \usepackage[parfill]{parskip} % Activate to begin paragraphs with an empty line rather than an indent

%%% PACKAGES
\usepackage{booktabs} % for much better looking tables
\usepackage{array} % for better arrays (eg matrices) in maths
\usepackage{paralist} % very flexible & customisable lists (eg. enumerate/itemize, etc.)
\usepackage{verbatim} % adds environment for commenting out blocks of text & for better verbatim
\usepackage{subfig} % make it possible to include more than one captioned figure/table in a single float
% These packages are all incorporated in the memoir class to one degree or another...

%%% HEADERS & FOOTERS
\usepackage{fancyhdr} % This should be set AFTER setting up the page geometry
\pagestyle{fancy} % options: empty , plain , fancy
\renewcommand{\headrulewidth}{0pt} % customise the layout...
\lhead{}\chead{}\rhead{}
\lfoot{}\cfoot{\thepage}\rfoot{}

%%% SECTION TITLE APPEARANCE
\usepackage{sectsty}
\allsectionsfont{\sffamily\mdseries\upshape} % (See the fntguide.pdf for font help)
% (This matches ConTeXt defaults)

%%% ToC (table of contents) APPEARANCE
\usepackage[nottoc,notlof,notlot]{tocbibind} % Put the bibliography in the ToC
\usepackage[titles,subfigure]{tocloft} % Alter the style of the Table of Contents
\renewcommand{\cftsecfont}{\rmfamily\mdseries\upshape}
\renewcommand{\cftsecpagefont}{\rmfamily\mdseries\upshape} % No bold!

%%% END Article customizations

%%% The "real" document content comes below...

\title{README\\ Architecture des fiches de lecture}
\author{Bounakoff, Duvernoy Frier}
%\date{} % Activate to display a given date or no date (if empty),
         % otherwise the current date is printed 

\begin{document}
\maketitle

\section{Note de mise a jour}
\subsection{2017-03-31}
R\'eunion avec Cyril. Apr\`es avoir beaucoup parl\'e de la bible haptique (BH) et des fiches de lecture, on en est arriv\'e \`a la conclusion tr\`es fructuante.\\
Plut\^ot que de s\'eparer les deux documents, on va tous les 3 partir sur un document unique, mais tr\`es riche.
Une premi\`ere partie d\'edi\'e \`a la Bible haptique avec l'ensemble des principes; suivi par toutes les fiches de lecture; suivi par la bibliographie des papiers sit\'ees dans son ensemble.\\


\section{Structure}
\subsection{Bible}
Chaque partie de la bible se verra argument\'e par une petite bibliographie \`a la fin. Les articles ainsi cit\'es en fin de partie qui dispose d'une fiche de lecture auront un lien direct pour acc\'eder \`a la fiche correspondante.\\


\subsection{Fiche de lecture}
Chaque article citera potentiellement d'autres articles de la meme fa\c con que le fait la BH. Les liens seront fait pour acc\'eder directement aux articles disposants d'une fiche.



\section{Architecture}
\subsection{Bible}
Chaque chapitre sera plac\'e sous la balise $\backslash$chapter

\subsection{Fiche de lecture}








\end{document}






