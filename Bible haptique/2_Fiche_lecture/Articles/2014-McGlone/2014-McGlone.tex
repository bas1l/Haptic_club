
\section {Discriminative and Affective Touch: Sensing and Feeling (2014) \textit{McGlone et al.} }
	Le papier pr\'{e}sente rapidement la diff\'{e}rence entre le toucher discriminatif et le toucher affectif.
	Le toucher discriminatif \'{e}tant le toucher qui nous permet de percevoir les charact\'{e}risitique d'une surface ou d'un objet (e.g.: texture, rigidit\'{e}, \ldots)
	Le toucher affectif est relatif au cot\'{e} \'{e}motionel transmis par le toucher (notamment à travers les caresses). 
	Le toucher discriminatif repose sur les 4 types de MRs de la peau glabres, tandis que le toucher affectif d\'{e}pend de fibre de type C pr\'{e}sente uniquement dans la peau pileuse.
	La suite du papier s'int\'{e}resse au sp\'{e}cificit\'{e} de cette fibre C et du role g\'{e}n\'{e}ral du toucher affectif dans le d\'{e}velopement humain.

	\subsection{Fibre C - Charact\'{e}ristique}
		La fibre C est \'{e}tonamment sensible aux caresses.
		En effet la sensibilit\'{e} de la fibre C est maximal pour des toucher faible (0.3-2.5mN), de vitesse lente (1-10cm/s) et des temperature proche de celle du corps humain (). 
		De ce fait, les caresses peuvent etre de plus ou moins bonne ``qualite'' selon si elle conforme a ces parametres.
		
		La fibre C est r\'{e}active a des stimuli mechanique ou de temperature, independemment, mais sera plus sensible si les deux stimuli sont present en mm temps.

		Il a \'{e}t\'{e} hypoth\'{e}tis\'{e} que le toucher discriminatif et le toucher affectif agirait comme un sens dual, où le premier permet de localiser le toucher et le deuxieme permet d'interepreter le toucher.


	\subsection{Toucher affectif et d\'{e}velopment}

	La peau est le premier organe en contact avec le monde exterieur et où le contact occure, du coup ça serait normal de penser que le toucher joue un role important dans la communication social"

	De maniere generale, le manque de toucher affectif augmente les risques de depression.

	Le toucher affectif permetrais de developper un futur comportement social.
	Si l'enfant a ete prive de toucher affectif, il ne developpera pas de comportement social dans le futur.
	Si l'enfant a ete stimule par un toucher affectif en etant enfant, il developpera un comportement social, et ce meme apres avoir ete coup\'{e} de tout contact pendant un certaine periode.

