\section{Mécanique du contact}

\subsection{Définition des termes}
Force : Dans la mécanique du point, une force est représenté par une direction (vecteur) et une amplitude (scalaire). Dans la mécanique du solide, le concept est étendu à une matrice 3x3 où chaque ligne représente les forces soumises sur les faces du cube élémentaire. On parle alors de tenseur de force. Les éléments sur la diagonale du tenseur sont appelés les efforts normaux, et les autres les efforts tangentiels. Dans certain cas, une matrice du moment peut être définie pour décrire les forces de rotations.\par
Déformation : La différence de géométrie entre l’état l’initiale et l’état qui suit l’exercions  du tenseur de force peut être définie avec un tenseur de déformation.\par
Élasticité : Décris la capacité d’un matériau de résister une force qui lui est appliqué. Exprimé avec le coefficient de Young.\par
Viscosité : Décris la composante temporelle qui lie application d’une force et déformation.\par
Compressibilité : Décris la capacité d’un matériau à se comprimer. Exprimé avec le coefficient de poisson.\par

….Distinction entre force normale, pression ; glissement, effort tangentiel; contraint, effort; tenseur de déformation…\par
…Distinction entre force et pression\par

\subsection{Contact statique}
Loi de hooke : Relie les deux grandeurs force et déformation\par
Contact Hertzien : Décrie les déformations\par

\subsection{Contact dynammque}
...

\subsection{Déformation local}
...

\subsection{Déformation distante -- Propagation des ondes}
Lors d’un stimulus tactile, des ondes se propagent à la surface de la peau, dans les couches inférieures de la peau et dans les organes. De ce fait les MRs distant du point de contact sont aussi stimulé par le contact. En plus de MRs cutanée, les MRs au niveau des muscles, tendons et articulation sont aussi stimulé. Il a été mis en évidence, que selon l’interaction tactile, le motif de propagation d’ondes sera différent. Ce qui alimentent l’hypothèse que ces ondes sont pris en comptent lors de l’interprétation d’un stimulus. Par exemple, lors d’un toucher d’exploration, la texture de la surface va créer un motif d’ondes particulier qui se fera sentir jusqu’au poignet, où des PC récepteurs pourront encodé la texture et ainsi permettre l’indentification de la texture parcouru.\par

Qu’est ce qui est pondérant, la propagation de la vibration : dans la peau, dans les os, dans les tendons ?\par


