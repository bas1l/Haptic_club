\section{Mécano-transduction}
Sous la stimulation, les mécanorécepteurs émettent des potentiels d’action le long des nerfs afférents et en direction du CNS. Le procédé par lequel une excitation sensorielle (ici mécanique) donne lieu à un potentiel d’action est nommé mécano-transduction. Ce procédé a fait l’objet de différentes études et a été modélisé.\par
Chaque mécanorécepteur encode la déformation mécanique en un potentiel d’action d’une manière différente. Il est a noté aussi que chaque mécanorécepteur semble sensible à différents types de stimuli (voir correspondant sous-partie).\par

\subsection{PC}
...

\subsection{RA}
...

\subsection{SAI}
...

\subsection{SAII}
...

\subsection{Fibre de type C}
...

\subsection{Exemple de modélisation}
...


