\section{Dimension du toucher}
Les EP, vu précédemment, servent notamment à renseigner sur la nature de l’objet/surface manipuler, qui sont caractérisé par les grandeurs physique et psychophysique suivante (Okamoto et al. 2013):
\begin{itemize}
	\item Texture;
	\begin{itemize}
		\item Rugueux/lisse;
		\begin{itemize}
			\item Macro-echelle (coarse);
			\item Micro -echelle (fine);
		\end{itemize}
		\item Dur/mou;
		\item Friction;
		\begin{itemize}
			\item Humide/sec;
			\item Glissant/collant;
		\end{itemize}
	\end{itemize}
	\item Température : Chaud/Froid;
	\item Forme : global/exact;
	\item Poids.
\end{itemize}
La texture regroupe plusieurs dimension et non seulement des information de rugosité. (voir liste au dessus).\par
Dualité de la perception de la rugosité : La rugosité d’une surface est perçu à deux échelles : Une macro-échelle, de l’ordre du millimetre et une micro-echelle de l’ordre du micrometre (voire nanometre). Cette difference d’échelle s’explique par les mecanisme du toucher qui encode la rugosité. La macro echelle vient du fait que les MR SA1 encode les informations de rugosité localement, là où peau et surface font contact. De ce fait, les limite de perceptions dépendent de l’acuité spatiale du sens du toucher. (threshold autour de 1mm). La micro-echelle dépend des MR PC qui encode les vibration créent par le doigt parcourant la surface. Ces vibrations se propagent dans tout le doigt, la paume des mains et vont même jusqu’au poignet. Des détails aussi fin que de qq centaine de nanomètre peuvent induire de telle vibration.\par
…Les parties les plus sensibles de notre corps humain sont les parties telles que : la figure, l’arrière de la nuque, les mains, le haut du bras, le torse, entre les jambes et la plante des pieds.\par
Et les changements d’état de celui-ci, glissement d’un objet, caresse,…\par
Le système somatosensoriel traite les données spatiotemporelles provenant des mécanorécepteurs [glissement, vibration], thermorécepteur [température] et nocicepteur [blessure] compris dans la peau.\par



