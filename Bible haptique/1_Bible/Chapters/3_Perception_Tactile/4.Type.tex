\section{Types de toucher}
Outre tous les caractéristiques biologiques de la peau et les informations qu’il est possible d‘en retirer quant à son importance dans la sensation d’une surface/d’un objet (i.e.: la souplesse/dureté de la surface de contact, sa température, sa texture, sa forme, etc.), il est intéressant de noter que notre perception se découpent en différents types de toucher, aux caractéristique différentes (exemple : récepteur stimulé). En tout on peut distinguer jusque quatre type de toucher : de manipulation, d’exploration, communicatif et protectif.\par

\subsection{Toucher de manipulation}
Type de peau : glabre\par

Le toucher de manipulation est spécifique à la peau glabre est aussi celui qui a été le plus étudié.

\subsection{Toucher d’exploration}
Type de peau : Glabre ; et Poilu pour la navigation\par
Le toucher d’exploration est…

\subsection{Toucher communicatif}
Type de peau : Poilu\par
Le toucher communicatif est …

\subsection{Toucher protectif}
Type de peau : Poilu\par
Le toucher est …




