\chapter{SENS, SENSATION, PERCEPTION}

\section{Les sens connus de l'être humain}
...9 sens chez l'homme, Aristote avait posé que l'Homme possédait 5 sens.
Définition d'une sens : ...

(On devrais essayer de les presenter dans l'ordre chronologique du dev du foetus)

\subsection{La vue}
...

\subsection{L'ouïe}
...

\subsection{Le toucher}
Le toucher ou somatoception est le sens transmis par la peau. Il est composé de la sensation des déformations de la peau mais aussi de la température et de la douleur. Le toucher existe aussi pour les organes internes.\par

\subsection{L'odorat}
...

\subsection{Le goût }
...

\subsection{Propriception}
...

\subsection{Equilibrioception}
...

\subsection{Nocioception}
...

\subsection{thermoception}
...

(Le sens haptique englobe-t-il thermo, nocio, proprio,...? Pt à éclaisir question de vocabulaire...)

\section{Distinction entre sensation et perception}
La sensation est la stimulation des récepteurs sensoriels qui produisent des potentiels d’actions que le cerveau interprètera. Une sensation se produit lorsqu’un organe sensoriel absorbe l’énergie provenant d’un stimulus physique venant de l’environnement extérieur. Les récepteurs sensoriels convertissent cette énergie physique en  potentiel d’action et les transmettent au cerveau via les nerfs.\par
La perception c’est quand le cerveau organise les informations sensorielle et les traduis/interprète en quelque chose significatif, qui a du sens ou qui peut être rationalisé. De plus, la perception est comment quelqu’un reçoit ce sentiment ou cette pensé, et y donne sens à travers la mémoire et l’émotion. La perception est principalement « comment » le cerveau interprète une sensation. L’information est obtenue à travers la collection, réception, transmission et mécanisme de codage.\par
Sensation et perception se complémentent l’un et l’autre pour donner du sens à notre expérience, mais restent deux complétement différente moyen de comment on interprète notre monde.\par
La perception est propre à chaque individu - différences intersujets, mais peut aussi varié au sein d’une même personne (i.e. des sensations identiques peuvent entraîner des perceptions différentes) - différences intrasujets. Ces différences viennent du fait que ce que nous percevons de notre environnement dépend de nos connaissances, de nos humeurs, de nos motivations, etc. Il arrive aussi qu’à certains moments, nos perceptions puissent dépasser les données sensorielles reçues, la perception va plus loin que la sensation. Exemple perception : un objet qu’on regarde peut nous apparaître chaud, rugueux, lourd.\par

\section{Cross-modalité et congruence}
Quand deux sensations venant d’un système sensoriel différent, arrivent au SNC, l’information est traitée comme un tout afin de donner lieu à une unique perception. On défini alors l’information provenant d’un système sensoriel comme une modalité et on dit que la perception résultante de plusieurs modalité est cross-modale.\par
Les informations provenant de deux modalité différentes peuvent être congruente (cohérente) et ainsi renforcé la perception. Dans le cas inverse les informations sont contradictoires et alors la perception sera biaisée vers l’une ou l’autre des modalités. Cette contradiction peut aussi donner lieu à des illusions (voir section correspondante).\par

\section{Illusion}
Illusion Haptique --- \url{https://lejournal.cnrs.fr/articles/lillusion-tactile-une-revolution-en-marche}\par

\subsection{ \textit{Fishbone illusion}}
Soit une surface découpé en trois partie : coté droit, milieu, coté gauche. Si les coté droit et gauche de la surface ont une texture plus rugueuse que le milieu, lorsque quelqu’un déplace sont doigt le long de la surface, il percevra le milieu de la surface à un renfoncement/creux.\par

\subsection{ \textit{Bump/hole illusion}}
Lorsque le doigt parcourt une surface, une information kinesthetic de creux ou de bosse, sera perçu de la même manière par une même personne.\par

\subsection{ \textit{Cutaneous Rabit illusion (saltation)}}
Original : Une série de courts pulses délivrés successivement à trois différente position sur la peau sont perçu comme un seul et unique stimulus bougeant progressivement le long de la peau. Comme si un petit lapin sautillée progressivement du premier au troisième stimulateur.\par
Réduit : 3 stimulation à 2 location, une des stimulations sera perçu au milieu des deux locations.\par
Exemple d’expérience :
\begin{itemize}
	\item Original : marche mieux avec 3-6 « tapements » et un intervalle inter-stimuli entre 20-250ms;
	\item Zone : 2.28cm2 sur l’index et 145.7cm2 sur l’avant bras.
\end{itemize}

\subsection{Illusion du peigne --- \textit{comb illusion}}
Lors d’une succession de stimuli de cisaillement, un stimulus de cisaillement plus « fort » peut induire l’illusion de la présence d’un pic.\par

\subsection{\textit{Apparent motion(effet phi)}}
L’illusion d’un déplacement continu peut être donné au travers de tapements successifs le long d’un trajet discontinu. La durée des tapements et l’intervalle inter-stimuli sont les paramètres principaux permettant la réalisation de cette expérience. Exemple d’expérience :
\begin{itemize}
	\item Tapement durée : 25-400ms;
	\item Meilleur avec tapement de 100ms et intervalle inter-stimulus de 70m;
	\item Intervalle 320ms avec 3 tapements et intervalle 20ms avec 12 tapements.
\end{itemize}

\subsection{Phantom-Funneling illusion}
Quand de bref stimuli sont presenté simultanément à différents points proche sur la peau, ils sont souvent perçus comme un unique stimulus central plutôt qu’une sensation phasique aux différentes positions. C’est comme si le stimulus tactile était « cheminée » vers une position centrale à laquelle le stimulus était perçu plus fort qu’à la position individuelle de stimulation.\par
Exemple d’expérience :
\begin{itemize}
	\item 3 stimulateurs sur l’avant bras (espacé de 30mm) à localisé dans une bande de 20mm autour du stimulateur du milieu. (robustesse: 80\%);
	\item 2 stimulateurs à point “phantom” au milieu, la position peut être variée vers un stimulateur ou l’autre en faisant varié le ratio d’amplitude des stimulateurs;
	\item Intensité au point « phantom »: $A_phantom^2= A_point1^2+A_point2^2$.
\end{itemize}

\subsection{Tau effect}
Le tau effect représente la dépendance temporelle de la perception des distances. Par exemple soit 3 stimuli successif sont présenté aux locations A,B et C. Si ces locations sont espacé tel que la distance AB soit 2 fois plus grande que la distance BC et que le temps entre le stimulus en A et celui en B soit deux fois plus lent qu’entre le stimulus B et C. Alors la distance AB sera perçu 4 fois plus grande que BC.\par
Cette illusion fonctionne aussi pour des stimuli mobile : Un stimulus rapide sera perçu comme parcourant une distance plus courte qu’un stimulus lent faisant le même parcourt.\par
Dans d’autre modalité, la dépendance spatial de la perception des durations (effet kappa) a été observé, mais ne semble pas existé pour le sens tactile.\par
Exemple d’expérience :
\begin{itemize}
	\item 3 points illusion : Meilleur si le ratio temps est inférieur à 4 :1, marche pour des distances allant de 30 à 85mm sur l’avant bras et des intervalle de temps inter-stimuli variant de 200 à 500ms;
	\item Stimulus mobile : Un stimulus rapide (2500mm/s) et un stimulus lent (10mm/s), la distance sera réduit de 50\%. Ne marche pas pour des stimuli de 50-200mm/s.
\end{itemize}



