\section{Distinction entre sensation et perception}
La sensation est la stimulation des récepteurs sensoriels qui produisent des potentiels d’actions que le cerveau interprètera. Une sensation se produit lorsqu’un organe sensoriel absorbe l’énergie provenant d’un stimulus physique venant de l’environnement extérieur. Les récepteurs sensoriels convertissent cette énergie physique en  potentiel d’action et les transmettent au cerveau via les nerfs.\par
La perception c’est quand le cerveau organise les informations sensorielle et les traduis/interprète en quelque chose significatif, qui a du sens ou qui peut être rationalisé. De plus, la perception est comment quelqu’un reçoit ce sentiment ou cette pensé, et y donne sens à travers la mémoire et l’émotion. La perception est principalement « comment » le cerveau interprète une sensation. L’information est obtenue à travers la collection, réception, transmission et mécanisme de codage.\par
Sensation et perception se complémentent l’un et l’autre pour donner du sens à notre expérience, mais restent deux complétement différente moyen de comment on interprète notre monde.\par
La perception est propre à chaque individu - différences intersujets, mais peut aussi varié au sein d’une même personne (i.e. des sensations identiques peuvent entraîner des perceptions différentes) - différences intrasujets. Ces différences viennent du fait que ce que nous percevons de notre environnement dépend de nos connaissances, de nos humeurs, de nos motivations, etc. Il arrive aussi qu’à certains moments, nos perceptions puissent dépasser les données sensorielles reçues, la perception va plus loin que la sensation. Exemple perception : un objet qu’on regarde peut nous apparaître chaud, rugueux, lourd.\par


