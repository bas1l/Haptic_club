\section{Cross-modalité et congruence}
Quand deux sensations venant d’un système sensoriel différent, arrivent au SNC, l’information est traitée comme un tout afin de donner lieu à une unique perception. On défini alors l’information provenant d’un système sensoriel comme une modalité et on dit que la perception résultante de plusieurs modalité est cross-modale.\par
Les informations provenant de deux modalité différentes peuvent être congruente (cohérente) et ainsi renforcé la perception. Dans le cas inverse les informations sont contradictoires et alors la perception sera biaisée vers l’une ou l’autre des modalités. Cette contradiction peut aussi donner lieu à des illusions (voir section correspondante).\par

