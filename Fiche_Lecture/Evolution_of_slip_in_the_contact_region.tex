\documentclass[../main.tex]{subfiles}

\begin{document}
Cette partie de l'article discute de comment la force de pression (normal et lateral), et donc le contact, evolue avant et pendant un glissement. 
Ces informations \'etant importante pour ajuster les forces avec lequels un objet est tenu ou manipul\'e. 
En effet ces forces augmentent lorsque un glissement est sur le point de se produire.
Les auteurs citent principalement l'article ``Effect of skin hydratation on the dynamics of fingertip gripping contact'',
où des images de glissement ont pu être enregistr\'e.
La principale observation est que lorsque la force tangentiel augmente, l'aire de contact r\'eduit petit à petit jusqu'à ce que le doigt entre en glissement.

Catttaneo [98] et Mindlin [99], utilisent les \'equation du contact de Hertz pour d\'ecrire la formation du ``slip annulus`` en fonction de la force de contact. 
Les \'equations de hertz permettent aussi de d\'eterminer que la force de contact est distribu\'e selon une parabole invers\'e avec un maximum au centre de la surface de contact (1.5p) et une limite de 0 au bords.
Cette application des \'equations de Hertz est justifi\'e pour des relativeent faibles forces (inf\'erieur à 1N).
(\'equations 4.1-4.3)

Tada et al. [96] et Andr\'e et al. [95] ont essay\'e de mettre en relation ces r\'esultat th\'eorique avec des mesures exp\'erimental. 
Ils ont trouv\'e que la th\'eories sousestiment les forces tangentielles mesur\'e exp\'erimentalement.
Ces r\'esultat exp\'erimentaux suggerent un seuil de valeur pour les forces tangentiel qui viendrais reduire le coefficient d'adh\'esion.

Tüzün et Walton [100] d\'etermine une valeur maximale pour les forces tangentielle. 
(\'equations 4.4 et 4.5)

Wang et Hayward [102] pointent les propri\'et\'e anisotropique du doigt et donc les limitent de l'application des \'equations de Hertz.
Par la suite l'article discutent de nouveaux model th\'eorique pour des forces sup\'erieur à 1N. 
(\'equations 4.6-4.8)
La conclusion etant que le coefficient d'adhesion diminue lin\'eairement avec les forces tangentielles.

La conclusion de cette partie suggere qu'une diminution du ''gross area`` pr\'ecedent toujours un glissement. 
Ceci pointe vers un mechanism de pelage\textbackslash de d\'ecolement.
N\'eanmoins, l'ensemble des \'etudes (et les \'equations \'etablis) considèrent seulement une valeur constant de la ''gross area``.

\end{document}

