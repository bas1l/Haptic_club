\documentclass[../main.tex]{subfiles}

\begin{document}
Cette partie de l'article discute de comment la force de pression (normal et lateral), et donc le contact, evolue durant un glissement. 
Les auteurs citent principalement l'article ``Effect of skin hydratation on the dynamics of fingertip gripping contact'',
où des images de glissement ont pu être enregistr\'e.
La principale observation est que lorsque la force tangentiel augmente, l'aire de contact r\'eduit petit à petit jusqu'à ce que le doigt entre en glissement.
Par la suite, les auteurs discutent des equations permettant de pr\'edir cette evolution et aussi le cas limite où le glissement d\'ebute.

\end{document}
