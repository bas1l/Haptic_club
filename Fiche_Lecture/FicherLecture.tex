\documentclass{article}
\title{Fiche Lecture}
\author{Bounakoff, Duvernoy, Frier}
\date{2016}

\usepackage[utf8]{inputenc}
\usepackage{subfiles}
\usepackage{blindtext}


\begin{document}
	\maketitle
	\section{Biom\'{e}canique}
	\subsection{Friction}
	\subsubsection {Finger pad friction and its role in grip and touch (2012) \textit{Adams et al.} }
	Le papier fait un \'{e}tat de l'art de la litt\'{e}rature sur la friction ayant lieu au point de contact entre doigt et surface.
	Le papier est divis\'{e} en 4 partie: Surface de contact, occlusion, l'\'{e}volution du slip dans la r\'{e}gion de contact et l'influnce de la vitesse de glissement.

	\paragraph {Evolution of Slip in the contact region}
	\subfile{Evolution_of_slip_in_the_contact_region}

	\paragraph {Occlusion}
	\subfile{Occlusion}

\end{document}

